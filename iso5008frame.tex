%	 DIN 5008 nach dem Stand vom April 2011.
%    A für Geschaeftsbriefe mit kleinem Briefkopf
%    B für Geschaeftsbriefe mit erweitertem Briefkopf

% Links:
% http://www.luk-korbmacher.de/Schule/Text/din676.htm 	<-- ref
% http://www.din5008.de/p0400140.htm					<-- genauer?
% http://www.helmol.de/din/03_briefb.htm				<-- DIN 5008-B-A4-IB
% http://www.druckeselbst.de/briefbogen/din676.php
% DIN 676 ???

\usepackage{picture,eso-pic,etoolbox,tikz} %xstring
\AddToShipoutPicture*{%
	\setlength\fboxsep{0pt}	%framebox border 0pt
	
	% === Switch-Defs  ===
	\newtoggle{printN}
	\newtoggle{printA}
	\newtoggle{printB}
	\newtoggle{printX} 
	\newtoggle{printT}

	% === Switches  ===	%	toggletrue	togglefalse | <<< Hier editieren >>>
	\toggletrue{printN} % zeigt Gemeinsamkeiten 
	\togglefalse{printA}	% zeige DIN 5008 A ?
	\toggletrue{printB}% zeige DIN 5008 B ?
	\toggletrue{printX}	% zeigt extras
	\togglefalse{printT}% Spielwiese
	
	% === Color-Defs  ===
	\newcommand \colorN {gray}		% normal / gemeinsam
	\newcommand \colorA {green}		% A
	\newcommand \colorB {cyan}		% B
	\newcommand \colorX {orange}	% extras
	\newcommand \colorT {red}			% Tests, obsolet
	
	% === Guards & Color  ===
	\newcommand \printN [1]{\ifhmode\unskip\fi\iftoggle{printN}{{\color{\colorN}{#1}}}{}\ignorespaces}
	\newcommand \printA [1]{\ifhmode\unskip\fi\iftoggle{printA}{{\color{\colorA}{#1}}}{}\ignorespaces}
	\newcommand \printB [1]{\ifhmode\unskip\fi\iftoggle{printB}{{\color{\colorB}{#1}}}{}\ignorespaces}
	\newcommand \printX [1]{\ifhmode\unskip\fi\iftoggle{printX}{{\color{\colorX}{#1}}}{}\ignorespaces}
	\newcommand \printT [1]{\ifhmode\unskip\fi\iftoggle{printT}{{\color{\colorT}{#1}}}{}\ignorespaces}
	
	% === Box-Defs ===
	\def \qBox(#1,#2)(#3,#4)[#5]#6{\put(#1,#2){\color{gray}\colorbox{#5!10}{\framebox(#3,#4){\parbox[t][#4][t]{#3}{#6}}}}} % 2 following rows with color break X coord.
  \def \boxN(#1,#2)(#3,#4)#5{\iftoggle{printN}{\qBox(#1,#2)(#3,#4)[\colorN]{#5}}{}}
	\def \boxA(#1,#2)(#3,#4)#5{\iftoggle{printA}{\qBox(#1,#2)(#3,#4)[\colorA]{#5}}{}}
	\def \boxB(#1,#2)(#3,#4)#5{\iftoggle{printB}{\qBox(#1,#2)(#3,#4)[\colorB]{#5}}{}}
	\def \boxX(#1,#2)(#3,#4)#5{\iftoggle{printX}{\qBox(#1,#2)(#3,#4)[\colorX]{#5}}{}}
	\def \boxT(#1,#2)(#3,#4)#5{\iftoggle{printT}{\qBox(#1,#2)(#3,#4)[\colorT]{#5}}{}}
	
	% === Marker-Defs ===
	\def \markerPriv(#1,#2){\put(#1,#2){\line(0,-1){2mm}}\put(#1,#2){\line(1,0){2mm}}}
	\def \markN(#1,#2){\ifhmode\unskip\fi\iftoggle{printN}{\color{\colorN}\markerPriv(#1,#2)}{}\ignorespaces} %norm
	\def \markA(#1,#2){\ifhmode\unskip\fi\iftoggle{printA}{\color{\colorA}\markerPriv(#1,#2)}{}\ignorespaces} %norm
	\def \markB(#1,#2){\ifhmode\unskip\fi\iftoggle{printB}{\color{\colorB}\markerPriv(#1,#2)}{}\ignorespaces} %norm
	\def \markC(#1,#2){\ifhmode\unskip\fi\iftoggle{printX}{\color{\colorX}\markerPriv(#1,#2)}{}\ignorespaces} %norm
	\def \markD(#1,#2){\ifhmode\unskip\fi\iftoggle{printT}{\color{\colorT}\markerPriv(#1,#2)}{}\ignorespaces} %norm
	 
	\AtTextCenter{%		Farb Legende
		\makebox(0,0)[c]{\rotatebox{45}{\color{gray}{
			\Huge \printN{DIN 5008}}%
			\printA{ A}%
			\printB{ B}%
			\printX{ X}%
			\printT{ T}%
		}}%
	}

  \AtPageUpperLeft{%
		\newcommand	\linksKopf {20mm}	% normaler Abstand links Kopf
		\newcommand \linksStd {25mm}	% normaler Abstand links
		% == Fluchtlinien ==
		%\printN{\multiput(\linksKopf,0)(0,-.01\paperheight){100}{\line(0,-1){.005\paperheight}}}	% Anschriftsfeld: 20 mm vom linken Blattrand	
		\printN{\multiput(\linksStd,0)(0,-.01\paperheight){100}{\line(0,-1){.005\paperheight}}}		% Fluchtlinie: 25 mm vom linken Blattrand
		\put(\paperwidth,0){% von rechts nach links negative vektoren verwenden
			\printN{\multiput(-10mm,0)(0,-.01\paperheight){100}{\line(0,-1){.005\paperheight}}}%	% Zeilenende: 10 mm vom rechten Blattrand
			\printX{\multiput(-\linksStd,0)(0,-.01\paperheight){100}{\line(0,-1){.005\paperheight}}}% sym. Bsp
		}%

		% == Briefkopf == [A] 27 mm bzw. [B] 45 mm vom oberen Blattrand
		\printA{\multiput(0,-27mm)(.01\paperwidth,0){100}{\line(1,0){.005\paperwidth}}}%
		\printB{\multiput(0,-45mm)(.01\paperwidth,0){100}{\line(1,0){.005\paperwidth}}}%	

		% == Anschrift ==
		\newcommand \anschrift {%
		 	Anschrift-1. Zeilen dritte Zeile ZV. o L\\
			2. Zeile	zweite Zeile Zusatzvermerke oder leer\\
			3. Zeile	erste Zeile Zusatzvermerke oder leer\\
			4. Zeile 	Name bzw. Firma\\
			5. Namenszusatz oder Str/Hausnummer\\
			6. Str/Hausnummer(Postfach) oder PLZ/Ort\\
			7. PLZ/Ort oder Länderangabe bzw. leer\\
			8. Länderangabe bzw. leer\\
		%	9. Zeile	leer
		}
			
		\boxA(\linksKopf,-33.9mm-40mm-5mm)(85mm,40mm){\anschrift}
		\boxA(\linksKopf,-33.9mm-5mm)(85mm,5mm){AnschriftAbsender}
		%\boxB(\linksKopf,-50.9mm-45mm)(85mm,45mm){}
		\boxB(\linksKopf,-50.9mm-40mm-5mm)(85mm,40mm){\anschrift}
		\boxB(\linksKopf,-50.9mm-5mm)(85mm,5mm){AnschriftAbsender}
		%8,46 mm Abstand zwischen dem Anschriftenfeld bzw. dem Informationsblock und den Leitwoertern der Bezugszeichenzeile (2 Leerzeilen)

		% ==Informationsblock==
		\boxA(\paperwidth-10mm-75mm,-33.9mm-40mm)(75mm,40mm){Informationsblock} % Informationsblock mindestens 40 mm hoch (variabel), 75 mm breit
		\boxB(\paperwidth-10mm-75mm,-50.8mm-40mm)(75mm,40mm){Informationsblock} % Informationsblock mindestens 40 mm hoch (variabel), 75 mm breit
		%\printA{\multiput(125mm,0)(0,-.01\paperheight){25}{\line(0,-1){.005\paperheight}}}		% Beginn Infoblock
		
		% ---------------- Marken (nach/ueber Boxen) ----------------		
		% == Kommunikationszeile / Absender, Datum etc. ==
		% -- Leitwoerter der Kommunikationszeile --
		\markA(\linksKopf,-63.5mm)
		\markB(\linksKopf,-80.4mm)
		% -- Text der Kommunikationszeile --
		\markA(\linksKopf,-67.7mm)
		\markB(\linksKopf,-84.7mm)
		
		% == Bezugszeichenzeile ==
		\markA(\linksStd,-80.4mm)	% Leitwoerter der Bezugszeichenzeile
		\markA(175mm,-80.4mm)	% Datum
		% Datum hier als Teil der Bezugszeichenzeile - sonst eine Zeile unter Empfaengerangaben als Teil des Infoblocks linksbuendig
		\printA{\put(175mm,-80.4mm-\baselineskip){\parbox[b][\baselineskip][t]{75mm}{Datum}}} 
		\markB(\linksStd,-97.4mm)	% Leitwoerter der Bezugszeichenzeile
		\markB(175mm,-97.4mm)	% Datum
		\printB{\put(175mm,-97.4mm-\baselineskip){\parbox[b][\baselineskip][t]{75mm}{Datum}}}
		% -- Text der Bezugszeichenzeile --
		\markA(\linksStd,-84.7mm)
		\markB(\linksStd,-101.6mm)
		
		% == Betreff == (bei einer Textzeile in der Bezugszeichenzeile)
		\markA(\linksStd,-97.4mm)
		\printA{\put(\linksStd,-97.4mm-5mm){\parbox[b][5mm][t]{30mm}{Betreff}}}
		\markB(\linksStd,-114.3mm)
		\printB{\put(\linksStd,-114.3mm-5mm){\parbox[b][5mm][t]{30mm}{Betreff}}}
		% Dem Betreffvermerk folgen zwei Leerzeilen.
		% Der Anrede folgt eine Leerzeile.	
		\printX{\multiput(0,-267mm)(.01\paperwidth,0){100}{\line(1,0){.005\paperwidth}}}% vorgeschlagene letzte Zeile
	}%DINmtext
  
	\AtPageLowerLeft{%
		\printN{\multiput(0,16.9mm)(.01\paperwidth,0){100}{\line(1,0){.005\paperwidth}}}% max letzte Zeile
	}
}
